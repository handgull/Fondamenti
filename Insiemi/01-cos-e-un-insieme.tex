\section{Cos'è un insieme}

Un \textbf{insieme} è una collezione non ordinata di oggetti distinti e ben definiti detti elementi dell'insieme.
Per convenzione gli insiemi sono denominati con una lettera maiuscola e sono delimitati da parentesi graffe, gli elementi sono indicati con una lettera minuscola. \\
Per ogni oggetto (anche un insieme) esistente è possibile chiedersi se esso \textbf{appartiene} o meno ad un determinato insieme. \\
Se un elemento appartiene ad A si scrive:
\begin{center}
    \textit{a} $\in$ A
\end{center}
Se un elemento \textit{b} non appartiene ad A si scrive:
\begin{center}
    \textit{b} $\notin$ A
\end{center}
L' \textbf{insieme universo} è l'insieme indicato con U che contiene tutti gli tutti gli elementi e tutti gli insiemi esistenti, compreso quindi anche se stesso. \\
L' \textbf{insieme vuoto}, ovvero l'insieme senza elementi, viene denotato con $\phi$. \\
Per ogni oggetto \textit{x}, esiste un insieme \{x\} che viene detto \textbf{singoletto}.
\begin{align*}
    &A = \{1, 2, 3\} \\
    &B = \{3, 2, 1\} \\
    &C = \{1, 1, 2, 3\}
\end{align*}
In questo caso abbiamo che A = B = C, dato che ordine e numerosità degli elementi non contano, come detto sopra. \\
\framebox{\{$\phi$\} non è l'insieme vuoto ma è un insieme (un singoletto) contenente l'insieme vuoto.}