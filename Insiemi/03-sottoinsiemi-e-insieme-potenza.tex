\section{Sottoinsiemi e insieme potenza}

\subsection{Sottoinsiemi di un insieme}
Consideriamo due insiemi:
\begin{align*}
    &A = \{1, 2, 3, 4\} \\
    &B = \{1, 2, 3\}
\end{align*}
Osserviamo che ogni elemento di B è anche elemento di A. In questo caso si dice che B è un sottoinsieme di A e si indica con la notazione
\begin{center}
    B $\subset$ A
\end{center}
La situazione può essere rappresentata tramite diagrammi di Venn:
\begin{center}
    \begin{tikzpicture}

        \begin{scope}[mytext/.style={text opacity=1,font=\large\bfseries}]

            \draw[draw = black] (0,0) circle (2);
            \draw[draw = black,name path=circle 2] (0,0) circle (1);

            \node[mytext] at (0,1.5) (A) {A};
            \node[mytext] at (0,0) (B) {B};
        \end{scope}
    \end{tikzpicture}
\end{center}
Per dire che un sottoinsieme B è contenuto o uguale ad A si può scrivere:
\begin{center}
    B $\subseteq$ A
\end{center}
\framebox{$\phi$ $\subseteq$ A $\forall$A} $\forall$ significa "per ogni" \\
Mentre per dire che B non è sottoinsieme di A possiamo scrivere:
\begin{center}
    $B \not \subset A$ o anche B $\not \subseteq$ A
\end{center}
Possiamo dire che per $\subseteq$ valgono le seguenti proprietà:
\begin{itemize}
    \item Riflessività: S $\subseteq$ S $\forall$S
    \item Transitività: se A $\subseteq$ B e B $\subseteq$ C allora A $\subseteq$ C
\end{itemize}
Se dati due insiemi C e D succede che C $\subseteq$ D e D $\subseteq$ C,
allora C è detto \textbf{sottinsieme improprio} di D. (C = D).\\
Ogni insieme (tranne l'insieme vuoto come vedremo a breve) accetta 2 sottoinsiemi impropri:
\begin{itemize}
    \item L'insieme stesso
    \item L'insieme vuoto
\end{itemize}
Se S $\subseteq$ T e S $\neq$ T allora diciamo che S è un \textbf{sottoinsieme proprio} di T
e che T è un \textbf{soprainsieme proprio} di S. \\
Repetita iuvant, scriviamo quello detto sopra in definizioni intensionali \\
S $\subset$ T = \{x : se x $\in$ S allora x $\in$ T con S $\neq$ T\} \\
(S = T) = \{x : x $\in$ S sse x $\in$ T\} \\
S $\subseteq$ T = \{x : S $\subset$ T oppure S = T\} \\

\subsection{Insieme potenza}
Un sottoinsieme di un insieme può essere chiamato \textit{parte}, l'insieme potenza
o \textbf{insieme delle parti} di A si indica con $\wp$(A) ed è l'insieme a cui appartengono tutti e soli i sottoinsiemi di A. \\
\begin{align*}
    \wp(S) &= \{X : X \subseteq S\} \\
    A &= \{1, 2\} \\
    \wp(A) &= \{\phi, \{1\}, \{2\}, \{1, 2\}\} \\
    \wp\phi &= \{\phi\}
\end{align*}
Se S è composto da n elementi (con n $\geq$ 0) il numero di elementi in $\wp$(S) è $2^n$.
Sapendo anche che la \textbf{cardinalità} di un insieme indica il numero di elementi di esso e si scrive:
\begin{center}
    A = \{1, 2, 3\} $|A|$ = 3
\end{center}
Allora potremmo anche dire che $\wp$(S) è $2^{|S|}$