\section{Operazioni fra insiemi}

\subsection{Intersezione di insiemi}
L'intersezione di due insiemi si scrive S $\cap$ T,
L'insieme risultante contiene tutti e soli gli elementi che appartenevano sia ad S che a T.
(naturalmente se S e T sono disgiunti S $\cap$ T = $\phi$). \\
S $\cap$ T = \{x : x $\in$ S e x $\in$ T\}
\begin{center}
    \begin{venndiagram2sets}
        \fillACapB
    \end{venndiagram2sets}
\end{center}
Per l'operazione $\cap$ valgono le seguenti proprietà: \\
\begin{itemize}
    \item Idempotenza: S $\cap$ S = S
    \item Commutatività: A $\cap$ B = B $\cap$ A
    \item Assorbimento: A $\cap$ B = A sse A $\subseteq$ B
    \item Associatività: (A $\cap$ B) $\cap$ C = A $\cap$ (B $\cap$ C)
\end{itemize}
Si noti inoltre che \framebox{A $\cap$ $\phi$ = $\phi$ $\forall$A}

\subsection{Unione di insiemi}
L'unione di due insiemi si scrive S $\cup$ T,
l'insieme risultante contiene tutti gli elementi di S e tutti quelli di T. \\
Definiamo S $\cup$ T = \{x : x $\in$ S oppure x $\in$ T\} \\
\begin{center}
    \begin{venndiagram2sets}
        \fillA \fillB
    \end{venndiagram2sets}
\end{center}
L'insieme unione come si può vedere è il più piccolo insieme che contiene sia A che B. \\
Per l'operazione $\cup$ valgono le seguenti proprietà: \\
\begin{itemize}
    \item Idempotenza: S $\cup$ S = S
    \item Commutatività: A $\cup$ B = B $\cup$ A
    \item Assorbimento: A $\cup$ B = A sse B $\subseteq$ A
    \item Associatività: (A $\cup$ B) $\cup$ C = A $\cup$ (B $\cup$ C)
\end{itemize}
Si noti inoltre che $\phi$ è l'elemento neutro \framebox{A $\cup$ $\phi$ = A} \\
Inoltre $\cup$ e $\cap$ sono legate da delle proprietà distibutive
\begin{itemize}
    \item A $\cup$ (B $\cap$ C) = (A $\cup$ B) $\cap$ (A $\cup$ C)
    \item A $\cap$ (B $\cup$ C) = (A $\cap$ B) $\cup$ (A $\cap$ C)
\end{itemize}
