\documentclass[oneside,12pt]{book}
\usepackage{amsmath}
\usepackage{venndiagram}
\usepackage{graphicx}
\usepackage{amsfonts}
\usepackage{tikz}
\usetikzlibrary{intersections}

\pagestyle{plain}

\title{Fondamenti dell'informatica}
\author{Andrea gullì \\ handgull}

\begin{document}

\maketitle

\tableofcontents

\chapter*{Introduzione}

Perchè studiare insiemi? La teoria degli insiemi è un fondamento della matematica e l'informatica deriva strettamente da essa \\
Concretamente parlando, il campo dell'informatica più influenzato dall'insiemistica
a mio avviso è quello delle \textbf{basi di dati}. \\
Ad esempio se una SELECT * FROM coinvolge più di una tabella verrà fatto il \textbf{prodotto cartesiano} tra le tuple\footnote{una tupla è un generico elemento di una relazione con attributi in un database relazionale.} delle tabelle del database. \\
Sempre nei database relazionali sono essenziali le operazioni di \textbf{unione}, \textbf{intersezione} (inner JOIN), di \textbf{differenza} e così via.

\input Insiemi/main.tex

\end{document}
